%% Copyright (C) 2009 by Tobias Elze
%% Journal of Vision LaTeX template version 1.0
%% This document may be used freely for your own submissions.

\documentclass{jov}

\begin{document}

\title{This is the title of the article}
\abstract{This document is an inofficial \LaTeX{} template to be used
to create a suitably formatted submission for the Journal of Vision. It
considers the changed layout requirements as of 2010. The complete
instructions for manuscript preparation are available at
\href{http://www.journalofvision.org/site/misc/ifora.xhtml}
{http://www.journalofvision.org/site/misc/ifora.xhtml}.
This is the text of the abstract. This version of the template is dated
\today. The template makes your \LaTeX{} manuscript look similar to 
Journal of Vision articles. However, in contrast to the two-column JOV
articles, manuscripts are in single colum mode.}

\author{Smith}{Jane Q.}
 {Schepens Eye Research Institute}
 {and Harvard Medical School, Boston, MA, USA}
 {http://}{jqs@schepens.harvard.edu}
\author{Elze}{Tobias}
 {Research Group Complex Structures in Biology and Cognition}
 {Max Planck Institute for Mathematics in the Sciences, Leipzig, Germany}
 {http://www.mis.mpg.de/complex/members/tobias-elze.html}
 {Tobias.Elze@mis.mpg.de}

\keywords{reverse correlation, triggered correlation, primate, recursive 
least squares, linear-nonlinear model, system identification}

\maketitle

\section{Introduction}

The jov class provides the following commands:
\begin{itemize}
\item \verb=\=title
\item \verb=\=abstract
\item \verb=\=author
\item \verb=\=keywords
\end{itemize}
To specify authors, use the \verb=\=author command for each author in the 
same order as they are to appear on the manuscript. The \verb=\=author 
command requires six arguments: Name, first name(s), institution, address, 
ehomepage, and email.

After specifying the commands above make use of \verb=\=maketitle to 
generate the title of the manuscript.

\section{To create a new manuscript}

To create a new manuscript for \textit{Journal of Vision}, do the 
following:
\begin{enumerate}
\item Open this template in your favorite \LaTeX{} editor.
\item Save the file under a new name.
\item Enter content into the appropriate sections, and make use of the 
standard \LaTeX{} commands.

\end{enumerate}

\section{Including references}

Treatment of cited material closely follows the Publication Manual of the 
American Psychological Association. The Bib\TeX{} file \verb=jovcite.bst= 
is a modified version of \verb=apacite.bst=. References can be included in 
rthe same way as required for \verb=apacite.bst=: 
\begin{itemize}
\item \verb=\citeA{AndrewsPollen1979}= $\to$ \citeA{AndrewsPollen1979}
\item \verb=\cite{RoordaWilliams2002}= $\to$ \cite{RoordaWilliams2002}
\item \verb=\citeNP{AndrewsPollen1979}= $\to$ \citeNP{AndrewsPollen1979}
\end{itemize}
The following Bib\TeX{} tags have been added to \verb=apacite.bst=: \verb=doi=, 
\verb=pubmedurl=, and \verb=articleurl=. See \verb=jovtemplate_latex.bib= 
for an example how to use a JOV bib file.

\section{Figures}

Add figures as usual in \LaTeX{}. Figure \ref{pdffigure} shows an example 
how to insert a figure stored in a pdf file.
\begin{figure}[h!]
\fbox{\parbox{\columnwidth}{Insert your figure here, for instance like 
this:\\[5mm]
\textbackslash{}includegraphics[width=\textbackslash{}columnwidth]\{file.pdf\}
\\[5mm] }}
\caption{Example how to add a figure stored in a pdf file}
\label{pdffigure}
\end{figure}

If you are using pdflatex, the following file types are processed by 
\verb=\includegraphics=: pdf, png, and jpg.

\section{Disclaimer}

Feel free to use this template on your own risk. Note that it is \textit{not} an 
official template and that it may contain errors or deviations from 
Journal of Vision's requirements. If you find errors or if you have 
suggestions for improvements, please contact me.\\ \\
\textcopyright{} 2009--2010, Tobias Elze.

\bibliography{jovtemplate_latex}
\bibliographystyle{jovcite}

\end{document}

