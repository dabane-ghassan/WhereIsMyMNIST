% !TEX root = DauceAlbigesPerrinet2020.tex
% !TEX encoding = UTF-8 Unicode
% -*- coding: UTF-8; -*-
% vim: set fenc=utf-8
% !TEX spellcheck = en-US
\section{Discussion} \label{sec:discussion}
\subsection{Summary}

In summary, we have proposed a visuo-motor action-selection model that implements a focal accuracy-seeking policy across the image. Our main modeling assumption here is an \emph{accuracy-driven} monitoring of action, stating in short that the ventral classification accuracy drives the dorsal selection on an accuracy map. The comparison of both accuracies amounts either to select a saccade or to keep the eye focused at the center, so as to identify the target. The predicted accuracy map has, in our case, the role of a value-based action selection map, as it is the case in model-free reinforcement learning. However, it also owns a probabilistic interpretation that may be combined with concurrent accuracy predictions (such as the one done through the ``What'' pathway) to explain more elaborate aspect of the decision making mprocesses which are relevant for visual search, such as the inhibition of return~\cite{Itti01}, without having to explictly implement such a heursitics. This combination of a scalar drive with action selection is reminiscent of the actor/critic principle proposed for long time in the reinforcement learning community~\cite{sutton1998reinforcement}. In biology, the ventral and the dorsolateral division of the striatum have been suggested to implement such an actor-critic separation~\cite{joel2002actor, takahashi2008silencing}. Consistently with those findings, our central accuracy drive and peripheral action selection map can respectively be considered as the ``critic'' and the ``actor'' of an accuracy-driven action selection scheme, with foveal identification/desambiguation taken as a ``visual reward''.

Moreover, one crucial aspect of vision highlighted by our model is the importance of centering objects in recognition. Despite the robust translation invariance observed on the ``What'' pathway, we found that there is a small tolerance radius of about $4$ pixels around the target's center that needs to be respected to maximize the classification accuracy. This relates to the idea of finding an absolute referential for an object, for which the recognition is easier. If the center of fixation is fixed, the log-polar encoding of an object has the notable properties to map object rotations and scalings toward translations in the radial and angular directions of the visual domain~\cite{Traver10}. The translation invariance found in convolutional processing may thus be extended to both rotation and scale invariance in the log-polar domain. Incorporating this scale and rotation invariance may thus extend the generalization capabilities of the model.

\subsection{Relation with other models}

{\color{magenta} Our model is, to our best knowledge, the first case of a bio-realistic log-polar implementations of an active vision framework.} We have thus provided a proof of concept that a log-polar encoding retina can efficiently serve object detection and identification over wide visual displays.

There are however a lot of models that reflect to some degree the biological principles of sequential visual processing.
First, active vision is of course an important topic in mainstream computer vision. In the case of image classification, it is considered as a way to improve object recognition by progressively increasing the definition over identified regions of interest, referred as ``recurrent attention''~\cite{mnih2014recurrent,fu2017look}.
{\color{magenta} \textbf{Rev 1} I'm not sure I understand why "recurrent attention is at odd with the functioning of biological systems" (line 565)}
Standing on a similar mathematical background, recurrent attention is however at odd with the functioning of biological systems, with a mere distant analogy with the retinal principles of foveal-surround visual definition.

Phenomenological bio-realistic models, such as the one proposed in Najemnik and Geisler's seminal paper~\cite{Najemnik05}, rely on a rough simplification, with foveal center-surround acuity modeled as a response curve. Despite providing a bio-realistic account of sequential visual search, the model owns no foveal image processing implementation. Stemming on Najemnik and Geisler's principles, a trainable center-surround processing system was proposed in~\cite{Butko2010infomax}, with a sequential scan of an image in a face-detection task, however the visual search task here relies  on a systematic scan over degraded image, with visual processing delegated to standard feature detectors.

Denil at al's paper~\cite{denil2012learning} is probably the one that shows the closest correspondence with our setup. It owns an identity pathway and a control pathway, in a What/Where fashion, just as ours. Interestingly, only the ``what'' pathway is neurally implemented using a random foveal/multi-fixation scan within the fixation zone. The ``Where'' pathway, in contrast, mainly implements object tracking, using  particle filtering with a separately learned generative process. The direction of gaze is here chosen so as to minimize the target position, speed and scale uncertainty, using the variance of the future beliefs as an uncertainty metric. The control part is thus much similar to a dynamic ROI tracking algorithm, with no direct correspondence with foveal visual search, or with the capability to recognize the target.

\subsection{Perspective}

Despite its simplicity, the generative model used to generate our visual display allowed to assess the effectiveness and robustness of our learning scheme, that should be extended to more complex displays and more realistic closed-loop setups. On the one side, the restricted $28\times28$ input used for the foveal processing is a mere placeholder, that should be replaced by more elaborate computer vision frameworks, such as Inception~\cite{szegedy2015going} or VGG-19~\cite{simonyan2014very}, that can handle a more ecological natural image classification. The main advantage of our peripheral image processing is its energy-efficiency. Our full log-polar processing pathway consistently conserves the high compression rate performed by retina and V1 encoding up to the action selection level. The organization of both the visual filters and the action maps in concentric log-polar elements, with radially exponentially growing spatial covering, can thus serve as a baseline for a future sub-linear (logarithmic) visual search in computer vision.
% {\color{red} \textbf{Rev 2} The authors mention “sub-linear (logarithmic) visual search”. As a reader I expected more than just a mention. How will this be achieved? This claim is not substantiated by any means.}
This may allow to detect an object in large visual environments at little cost, which should be particularly beneficial when the computing resources are under constraint, such as for drones or mobile robots.

Finally, our model relies on a strong idealization, assuming the presence of a unique target. This is well adapted to a fast changing visual scene as is demonstarted by our ability to perform as fast as 5 saccades per second to detect faces in a cluttered environment~\cite{Martin18}. However, some visual scenes ---such as when looking at a painting in a museum--- allow for a longer inspection of its details.  The presence of many targets in a scene should be addressed, which amounts to sequentially select targets, in combination with implementing a more elaborate inhibition of return mechanism to account for the trace of the performed saccades. This would generate more realistic visual scan-paths over images. Actual visual scan path over images could also be used to provide priors over action selection maps that should improve realism.  Identified regions of interest may then be compared with the baseline bottom-up approaches, such as the low-level feature-based saliency maps~\cite{Itti01}. Maximizing the Information Gain over multiple targets needs to be envisioned with a more refined probabilistic framework extending previous models~\cite{Friston12}, which would include phenomena such as mutual exclusion over overt and covert targets. How the brain may combine and integrate these various probabilities is still an open question, that amounts to the fundamental binding problem.
