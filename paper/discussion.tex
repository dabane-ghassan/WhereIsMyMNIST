% !TEX root = plos-paper.tex
% !TEX encoding = UTF-8 Unicode
% -*- coding: UTF-8; -*-
% vim: set fenc=utf-8
% !TEX spellcheck = en-US

\section*{Discussion} \label{sec:discussion}
%\subsection{Summary}
%## Main results:

In summary, we have proposed a visuo-motor action-selection model that implements a focal accuracy-seeking policy across the image. It relies on an interpretation of the Information Gain metric as a difference between central and peripheral accuracy processing. Each accuracy is predicted through separate processing pathways, namely the ``What'' pathway for the central pixels and the ``Where'' pathway for the periphery.  The comparison of both accuracies amounts either to select a saccade or to keep the eye focused at the center, so as to identify the digit. The predicted accuracy map has, in our case, the role of a value-based action selection map, as it is the case in model-free reinforcement learning. However, it also owns a probabilistic interpretation that may be combined with concurrent accuracy predictions (such as the one done through the ``what'' pathway) to bring out more elaborate decision making which are relevant for visual search, such as the inhibition of return \cite{Itti01}. 

Our main modeling assumption here is an \emph{accuracy-driven} monitoring of action, stating in short that the ventral classification accuracy drives the dorsal selection on an action map. This combination of a scalar drive with action selection is reminiscent of the actor/critic principle proposed for long time in the reinforcement learning community \cite{sutton1998reinforcement}. In biology, the ventral and the dorsolateral division of the striatum have been suggested to implement such an actor-critic separation \cite{joel2002actor,takahashi2008silencing}. Consistently with those findings, our central accuracy drive and peripheral action selection map can respectively be considered as the ``critic'' and the ``actor'' of an accuracy-driven reinforcement learning scheme, with foveal identification/desambiguation taken as a ``visual reward''.

Moreover, one crucial aspect of vision highlighted by our model is the importance of centering objects in recognition. Despite the robust translation invariance observed on the ``What'' pathway, there is small radius of 2-3 pixels around the target's center that needs to be respected to maximize the classification accuracy. This relates to the idea of finding an absolute referential for an object, for which the recognition is easier. If the center of fixation is fixed, the log-polar encoding of an object has the notable properties to map object rotations and scalings toward translations in the radial and angular directions of the visual domain \cite{Traver10}. The translation invariance found in convolutional processing may thus be extended to both rotation and scale invariance in the log-polar domain. Incorporating this scale and rotation invariance may thus extend the generalization capabilities of the model.



Despite its simplicity, the generative model used to generate our visual display allowed to assess the effectiveness and robustness of our learning scheme, that should be extended to more complex displays and more realistic closed-loop setups.
On the one side, the restricted 28$\times$28 input used for the foveal processing is a mere placeholder, that should be replaced by more elaborate image processing frameworks
\cite{simonyan2014very}, that can deal with natural image classification. 
%By preserving a probabilistic interpretation in bio-realistic action selection, 
On the other side, the main advantage of our peripheral image processing is its  energy-efficiency. Our full log-polar processing pathway consistently conserves the high compression rate performed by retina and V1 encoding up to the action selection level. The organization of both the visual filters and the action maps in concentric log-polar elements, with radially exponentially growing spatial covering, can thus serve as a baseline for a future sub-linear (logarithmic) visual search in computer vision. This may allow to detect an object in large visual environments at little cost, which should be particularly beneficial when the computing resources are under constraint, such as for drones or mobile robots. 


Finally, our model relies on a strong idealization, assuming the presence of an unique target. The presence of many targets in a scene should be addressed, which amounts to sequentially select targets, in combination with implementing an inhibition of return mechanism. 
%Moreover, inhibition of return mechanism could envisioned by preserving a probabilistic interpretation in bio-realistic action selection. 
This would generate more realistic visual scan-paths over images. %This could be used to provide realistic priors over action selection maps.  
%In particular, identified regions of interest may then be compared with the baseline bottom-up approaches, such as the low-level feature-based saliency maps~\cite{Itti01}. 
Actual visual scan path over images could also be used to provide priors over action selection maps that should improve realism.  %
%It may indeed be possible to consider , and 
Identified regions of interest may then be compared with the baseline bottom-up approaches, such as the low-level feature-based saliency maps~\cite{Itti01}. 
Maximizing the Information Gain over multiple targets needs to be envisioned with a more refined probabilistic framework, including mutual exclusion over overt and covert targets. How the brain may combine and integrate these various probabilities is still an open question, that amounts to the fundamental binding problem. %: How is it possible to meaningfully combine independently extracted features.

%Implemented in combination with multi-target object  should be quantitatively compared with the linear search time on larger displays. time necessary to scan all positions on a regular grid), 