% !TEX root = paper.tex
% !TEX encoding = UTF-8 Unicode
% -*- coding: UTF-8; -*-
% vim: set fenc=utf-8
% !TEX spellcheck = en-US
In computer vision, the visual search task consists in extracting a scarce and specific visual information (the ``target'') from a large and crowded visual display. This task is usually implemented by scanning the different possible target identities at all possible spatial positions, hence with strong computational load. The human visual system employs a different strategy, combining a foveated sensor with the capacity to rapidly move the center of fixation using saccades. Saccade-based visual exploration can be idealized as an inference process, assuming that the target position and category are independently drawn from a common generative measure process. Knowing that process, visual processing is then separated in two specialized pathways, the ``where'' pathway mainly conveying information about target position in peripheral space, and the ``what'' pathway mainly conveying information about the category of the target. We consider here a dual neural network architecture learning independently where to look and then at what to see. This allows in particular to infer target position in retinotopic coordinates, independently to its category. This framework was tested on a simple task of finding digits in a large, cluttered image. Simulation results demonstrate the benefit of specifically learning where to look before actually knowing the target category. The approach is also energy-efficient as it includes the strong compression rate performed at the sensor level, by retina and V1 encoding, which is preserved up to the action selection level, highlighting the advantages of bio-mimetic strategies with regards to traditional computer vision when computing resources are at stake.
% TODO : include whenever we do more than one saccade...
% Without a saccade, the accuracy drops to the baseline at half the width of the target from the center of fixation, while actuating a saccade is beneficial in up to 3 times its size, allowing a much wider covering of the image. The ratio between the marginal accuracies shows that this model is computationally an order of magnitude more efficient than that of a classical brute-force framework. Until the foveal classifier is confident, the system should thus perform saccades to the most likely target position. The different accuracy predictions, such as the ones done in the ``what'' and the ``where'' pathway, may also explain more elaborate decision making, such as the inhibition of return.
%This provides evidence of the importance of identifying ``putative interesting targets'' first and we highlight some possible extensions of our model both in computer vision and modeling.
% TODO: We compared the results of this model with classical psychophysical results in visual search
%This generic visual search problem is of broad interest to machine learning, computer vision and robotics, but also to neuroscience, as it speaks to the mechanisms underlying foveation and more generally to low-level attention mechanisms. From a computer vision perspective, the problem is generally addressed by processing the different hypothesis (categories) at all possible spatial configuration through dedicated parallel hardware. The human visual system, however, seems to employ a different strategy, through a combination of a foveated sensor with the capacity of rapidly moving the center of fixation using saccades.
%Visual processing is done through fast and specialized pathways, one of which mainly conveying information about target position and speed in the peripheral space (the "where" pathway), the other mainly conveying  information about the identity of the target (the "what" pathway). The combination of the two pathways is expected to provide most of the useful knowledge about the external visual scene. Still, it is unknown why such a separation exists.
