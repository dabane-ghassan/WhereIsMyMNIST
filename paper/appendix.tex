\section*{General case: Visual information gain maximization}\label{sec:case1}

Consider a view $\boldsymbol{x}$ generated from a target $\boldsymbol{y}$ viewed at retinocentric position $\boldsymbol{u}$.

Consider first that :
\begin{itemize}
	\item The generative model  $p(X|\boldsymbol{y}, \boldsymbol{u})$ is known
	\item The retinocentric position  $\boldsymbol{u}$ is known.
	\item The view $\boldsymbol{x}$ is known.
	\item The target category  $\boldsymbol{y}$ is unknown.
\end{itemize}



The question comes how to choose the new retinocentric position $\boldsymbol{u}'$ in order to maximize the \emph{mutual information} between $\boldsymbol{x}|\boldsymbol{u}$ (current view) and $\boldsymbol{x}'|\boldsymbol{u}'$ (future view).

In general, the visual Information Gain between two visual fields $\boldsymbol{x}|\boldsymbol{u}$  and $\boldsymbol{x}'|\boldsymbol{u}'$ is:

\begin{align*}
\text{IG}(\boldsymbol{x}|\boldsymbol{u}; \boldsymbol{x}'| \boldsymbol{u}')
&= -\log p(\boldsymbol{x}|\boldsymbol{u})
+ \log p(\boldsymbol{x}|\boldsymbol{u}, \boldsymbol{x}', \boldsymbol{u}')
\end{align*}

\paragraph{Information Gain Lower Bound}
Consider now that given  $\boldsymbol{x}$ and $\boldsymbol{u}$, the target category  $\boldsymbol{y}$ can be \emph{inferred} using Bayes rule, i.e.:
$$ P(Y|\boldsymbol{x}, \boldsymbol{u}) \propto  P(\boldsymbol{x}|Y, \boldsymbol{u}) $$
Then, it can be shown (see \cite{dauce2018}) that :
$$\text{IG}(\boldsymbol{x}|\boldsymbol{u}; \boldsymbol{x}'| \boldsymbol{u}') \geq \mathbb{E}_{\boldsymbol{y}\sim p(Y|\boldsymbol{x}, \boldsymbol{u})} \left[\log p(\boldsymbol{y}|\boldsymbol{x}', \boldsymbol{u}') - \log(\pi(\boldsymbol{y})) \right]$$
with  $\pi(\boldsymbol{y})$ the prior over the $\boldsymbol{y}$'s .
When the prior is uniform, the information gain lower bound (IGLB) simplifies to $\mathbb{E}_{\boldsymbol{y}\sim p(Y|\boldsymbol{x}, \boldsymbol{u})} \left[\log p(\boldsymbol{y}|\boldsymbol{x}', \boldsymbol{u}')\right] + c$, with $c$ a constant.

\paragraph{Predictive approach}
One can adopt a \emph{predictive} approach to choose the new eye orientation $\boldsymbol{e}'$:
\begin{itemize}
	\item First choose a new retinocentric position $\boldsymbol{u}'$ that will maximize the  information gain.
	\item Then choose $\boldsymbol{e}'$ such that $$\boldsymbol{z} - \boldsymbol{e}' = \boldsymbol{u}'$$ i.e. $$\boldsymbol{e}' = \boldsymbol{e} + \boldsymbol{u} - \boldsymbol{u}'$$
\end{itemize}

The predictive approach needs three predictive steps:
\begin{itemize}
	\item $p(Y|\boldsymbol{x}, \boldsymbol{u})$ is the current posterior over the target category inferred from the current observation,
	\item $\boldsymbol{x}'\sim p(X|\boldsymbol{y},\boldsymbol{u}')$ is the predicted view generated by the model assuming that the target $\boldsymbol{y}$ is seen from from $\boldsymbol{u}'$,
	\item and $p(\boldsymbol{y}|\boldsymbol{x}', \boldsymbol{u}')$ is the predicted posterior for   assumption $\boldsymbol{y}$, given $\boldsymbol{x}'$ and $\boldsymbol{u}'$.
\end{itemize}

Then the optimal new retinocentric position is:
\begin{align*}
\hat{\boldsymbol{u}}' &= \underset{\boldsymbol{u}' }{\text{ argmax }}
 \mathbb{E}_{\boldsymbol{y}\sim p(Y|\boldsymbol{x}, \boldsymbol{u})}
 \left[\mathbb{E}_{ \boldsymbol{x}' \sim p(X|\boldsymbol{y}, \boldsymbol{u}')}
 \left[\log p(\boldsymbol{y}|\boldsymbol{x}', \boldsymbol{u}')\right]\right]\\
  %&= \underset{\boldsymbol{u}' \in \mathcal{U}}{\text{ argmax }} A(\boldsymbol{u}'|\boldsymbol{x}, \boldsymbol{u})
\end{align*}

Taking $\delta \boldsymbol{e} = \boldsymbol{u} - \boldsymbol{u}'$, the optimal eye displacement is:
\begin{align*}
\widehat{\delta\boldsymbol{e}} &= \underset{\delta\boldsymbol{e} }{\text{ argmax }}
\mathbb{E}_{\boldsymbol{y}\sim p(Y|\boldsymbol{x}, \boldsymbol{u})}
\left[\mathbb{E}_{ \boldsymbol{x}' \sim p(X|\boldsymbol{y}, \boldsymbol{u}- \delta \boldsymbol{e})}
\left[\log p(\boldsymbol{y}|\boldsymbol{x}', \boldsymbol{u}-\delta\boldsymbol{e})\right]\right]\\
%&= \underset{\delta\boldsymbol{e}}{\text{ argmax }} A(\delta\boldsymbol{e}|\boldsymbol{x}, \boldsymbol{u})
\end{align*}

%For each possible target category $\boldsymbol{y}$, $A_{\boldsymbol{y}}(\boldsymbol{u}') = \mathbb{E}_{ \boldsymbol{x}' \sim p(X|\boldsymbol{y}, \boldsymbol{u}')}
%\left[\log p(\boldsymbol{y}|\boldsymbol{x}', \boldsymbol{u}')\right]$ is the \emph{class-specific log posterior} map and




%Then:
%$$\tilde{A}_{\boldsymbol{y}}(\boldsymbol{u}') = \mathbb{E}_{ \boldsymbol{x}' \sim p(X|\boldsymbol{y}, \boldsymbol{u}')}$$

\emph{(TODO : Attention il faudrait à partir de maintenant une carte qui moyenne les log posteriors car l'espérance du log n'est pas  égale au log de l'espérance, i.e. $r_{\theta}^{\text{log}}(\boldsymbol{u}|q) = \mathbb{E}_{\boldsymbol{y}\sim q(Y)} \left[\mathbb{E}_{ \boldsymbol{x} \sim p(X|\boldsymbol{y}, \boldsymbol{u})}  \log p_\theta(\boldsymbol{y}|\boldsymbol{x}, \boldsymbol{u}) \right]$).}
\newline
